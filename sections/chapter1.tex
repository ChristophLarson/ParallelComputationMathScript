%\chapterimage{chapter_head_2.pdf} % Chapter heading image

%----------------------------------------------------------------------------------------
\chapter{Introduction C++}
%----------------------------------------------------------------------------------------
This chapter gives a brief introduction to the core features of the C++ language. However, the focus on this chapter is to introduce the C++ standard template library, see Section~\ref{chapter:stl}, which is essential to implement mathematical equations and algorithms. Therefore, we quickly look into these features as we need them to implement the numerical examples in Part~\ref{part:numerical:examples}. For more details we refer to
\begin{itemize}
\item \fullcite{andrew2000accelerated}
\end{itemize}
since this book gives an excellent pragmatic overview with a lot of examples. For even more C++ basics, we refer to
\begin{itemize}
\item \fullcite{stroustrup2014programming}.
\end{itemize}
%----------------------------------------------------------------------------------------
\section{History of C and C++}
%----------------------------------------------------------------------------------------

%----------------------------------------------------------------------------------------
\section{Getting started with C++}
%----------------------------------------------------------------------------------------
To begin with C++ programming, we look at a simple C++ program, the so-called ``Hello World'' example which most programming language start with. Listing~\ref{code:hello:world} shows this program and the first line in green a comment is shown. A  single-line comments always starts with \lstinline[language=C++]|//| and is used to explain the functionality of the program or the next lines of codes. It is also possible to use multi-line comments\endnote{\url{https://en.cppreference.com/w/cpp/comment}} by using \lstinline[language=C++]{/* */}. Comments are important to understand the program, especially if the code is shared with other collaborators. Fore more details we refer to~\cite{kernighan1974elements}.

The second line starts with a so-called include directive\endnote{\url{https://en.cppreference.com/w/cpp/preprocessor/include}} \lstinline[language=C++]{#include <iostream>}. This include directive is needed to include functionality of the C++ standard library, see Chpater~\ref{chapter:stl}, in our case we include the \lstinline|iostream| header to print the Hello World to the terminal, see Line 6.

The fourth line \lstinline[language=C++]{int main()} starts with the so-called Main function\endnote{\url{https://en.cppreference.com/w/cpp/language/main_function}} which is the entry point of the program and all code lines within are executed sequentially one by one. Every C++ which will be compiled to an executable needs exact one function called \lstinline[language=C++]|main| which have a integer \lstinline[language=C++]{int} as its return type. On most operation systems a return value of zero means that the program executed successfully and any other value indicates an failure. The second last line \lstinline[]|return| is the so--called return statement\endnote{\url{https://en.cppreference.com/w/cpp/language/return}} which has to match the return type in front of the \lstinline[language=C++]{int main()}.


\lstinputlisting[language=C++,caption={A simple C++ program, the so-called ``Hello World'' example, which most languages start with.\label{code:hello:world}},float,floatplacement=h]{ParallelComputationMathExamples/chapter2/lecture1-main.cpp}

Once we have written the program, we have to compile the C++ code into an executable to run the code and print ``Hello world'' to the terminal. Note that there are plenty of C++ compilers\endnote{\url{https://en.wikipedia.org/wiki/List_of_compilers\#C++_compilers}} available, however this book will use the GNU Compiler Collection (GCC) for all examples. Line 1 in Listing~\ref{code:hello:world:compile} shows how to compile the file \lstinline[language=bash]|lecture1-1.cpp|, which contains the C++ code in Listing~\ref{code:hello:world}, to an executable. The GCC provides the \lstinline[language=bash]{g++} compiler to compile C++ code and the \lstinline[language=bash]{gcc} compiler to compile C code. As the first argument to the \lstinline[language=bash]{g++} compiler the file name of the C++ is provided and with the \lstinline[language=bash]|-o| option the name of the executable is specified. To run the generated executable, we type \lstinline[language=bash]|./lecture-1-1| in the terminal. Note for the basic usage of the Linux terminal we refer to~\cite{newham2005learning,robbins2016bash}.

\lstinputlisting[language=bash,caption={Compilation and execution of the C++ program.\label{code:hello:world:compile}},float,floatplacement=h,firstline=2, lastline=3]{ParallelComputationMathExamples/chapter2/lecture1-main.sh}

\begin{exercise}
Download the example program\endnote{\url{https://github.com/diehlpkteaching/ParallelComputationMathExamples}} from GitHub and compile it with your favorite C++ compiler. After you ran the example you could try to modify it, for example you could print a different text or add a second line to the output.
\end{exercise}


%----------------------------------------------------------------------------------------
\section{Fundamental data types}
%----------------------------------------------------------------------------------------
\index{data types!fundamental}
In this section the fundamental data types\endnote{\url{https://en.cppreference.com/w/cpp/language/types}} provided by the C++ language are introduced. First, the numeric data types are introduced. To represent natural numbers $\mathbb{N}=\{0,1,2,\ldots \}$ the \lstinline[language=C++]|unsigned int| data type is available. To represent integer numbers $\mathbb{Z}=\{\ldots,-2,-1,0,1,2,\ldots \}$ the \lstinline[language=C++]|int| data type is available. For all these data types following options: \lstinline[language=C++]|short|, \lstinline[language=C++]|long|, and \lstinline[language=C++]|long long| are available. For more details about the binary numeral system we refer to~\cite{gilli1965binary}. To represent real numbers $\mathbb{R}$ the \lstinline[language=C++]|float| data type and \lstinline[language=C++]|double| data type are available. Fore more details about the IEEE 474 standard how floating point numbers are represented in the computer we refer to~\cite{4610935,goldberg1991every}. Table~\ref{chapter2:table:datatypes} summarizes all the available numeric data types and there ranges. The next section shows how to get the range of the IEEE 474 standard how floating point numbers.

\begin{table}[h]
\centering
\begin{tabular}{lccc}
\toprule
Data type & Size (Bytes) & Min & Max \\\midrule
\multicolumn{4}{c}{Natural numbers $\mathbb{N}$ }\\\midrule
\lstinline[language=C++]|unsigned short int| & 2 & 0 & 65,535  \\ 
\lstinline[language=C++]|unsigned int| & 4 & 0 & 4,294,967,295 \\ 
\lstinline[language=C++]|unsigned long int| & 4 & 0 & 4,294,967,295 \\ 
\lstinline[language=C++]|unsigned long long int| & 8 & 0 & 8,446,744,073,709,551,615 \\ \midrule
\multicolumn{4}{c}{Integer numbers $\mathbb{Z}$ }\\\midrule
\lstinline[language=C++]|short int| & 2 & -32,768 & 32,768 \\
\lstinline[language=C++]|int| & 4 & -2,147,483,648 & 2,147,483,648 \\
\lstinline[language=C++]|long long int| & 8 & $-2^{63}$ & $2^{63}-1$ \\\midrule
\multicolumn{4}{c}{Real numbers $\mathbb{R}$ }\\\midrule
\lstinline[language=C++]|float| & 4 &  &  \\
\lstinline[language=C++]|double| & 8 &  &  \\
\bottomrule
\end{tabular} 
\caption{Overview of the fundamental numeric data types.}
\label{chapter2:table:datatypes}
\end{table}

To represent a boolean value $\mathbf{B}=\{0,1\}$ the \lstinline[language=C++]|bool| data type which has either one of the two values \lstinline[language=C++]|true| or \lstinline[language=C++]|false|. Note that the C++ STL offers \lstinline|std::complex|\endnote{\url{https://en.cppreference.com/w/cpp/numeric/complex}} for complex number $\mathbb{C}$, however, this one is not within the fundamental data types.

%----------------------------------------------------------------------------------------
\section{Statements and flow control}
%----------------------------------------------------------------------------------------

%----------------------------------------------------------------------------------------
\subsection{Iteration statements}
%----------------------------------------------------------------------------------------
\index{statements!iteration} For some applications, we have to repeat some operations multiple times. The C++ language provides the so-called iteration statements. There are two iteration statements, the \lstinline[language=C++]|for| loop and the \lstinline[language=C++]|while| loop, respectively. Let us look how to compute the sum of the numbers from one up ton $n$
\begin{align}
r = \sum\limits_{i=1}^n i\text{.}
\end{align}
The first option is using a \lstinline[language=C++]|for| loop statement\endnote{\url{https://en.cppreference.com/w/cpp/language/for}}\index{loop!for statement} which shown in Listing~\ref{code:for:loop}. Line 9 shows the \lstinline[language=C++]|for| loop statement with its three arguments. First, the so--called loop variable \lstinline[language=C++]{size_t i = 0} which is initialized to zero. Note that the loop variable is only defined within the loop's body (The part between the curly braces). Second, the so--called condition statement \lstinline[language=C++]{i < n} which means that the loop body is repeated until the variable $i$ is larger than $n$. The third statement manipulates the loop variable, in our case the loop variable is incremented by one after the loop body was executed once. Note that we use the \lstinline[language=C++]|for| loop statement, if we know how often we have to loop through the loop body in advance.


\lstinputlisting[language=C++,caption={Computation of the sum from one up to $n$ using the for loop statement. \label{code:for:loop}},float,floatplacement=h]{ParallelComputationMathExamples/chapter2/lecture1-for.cpp}

The second option is using a \lstinline[language=C++]|while| loop statement\endnote{\url{https://en.cppreference.com/w/cpp/language/while}}\index{loop!while statement} which shown in Listing~\ref{code:while:loop}. Line 10 shows the \lstinline[language=C++]|while| loop statement with its one argument. This is the so-called condition statement  \lstinline[language=C++]{i < n} which means that the loop body is repeated until the variable $i$ is larger than $n$. Note in the previous example we had three arguments. Here, the loop variable is declared before the loop in Line 9 and the third statement is  in Line 13 where the loop variable is incremented by one in each iteration. Note that we use the \lstinline[language=C++]|while| loop statement, if we do not know the amount of iterations in advance. This example has shown that we can write every \lstinline[language=C++]|for| loop statement as a \lstinline[language=C++]|while| loop statement. For more details we refer to~\cite[Chapter~2]{andrew2000accelerated}.   


\lstinputlisting[language=C++,caption={Computation of the sum from one up to $n$ using the while loop statement..\label{code:while:loop}},float,floatplacement=h]{ParallelComputationMathExamples/chapter2/lecture1-while.cpp}

\begin{exercise}
Define in your own words in which case you should use a \lstinline[language=C++]|for| loop statement and a \lstinline[language=C++]|while| loop statement. 
\end{exercise}

%----------------------------------------------------------------------------------------
\subsection{Selection statements}
%----------------------------------------------------------------------------------------
\index{statements!selection}
For some applications, we have to select different behavior of the code depending on conditions. Equation~\ref{eq:chapter2:if} shows how to compute the sum from one to $n$ with different cases for even and odd numbers, If the number is even the number keeps the same and for odd numbers the number is squared.
\begin{align}
r = \sum\limits_{i=1}^n f(i) \text{  with  } f(i) = 
\begin{cases}
i, \text{ if } i \text{ is even} \\
i^2, \text{ else}
\end{cases}
\label{eq:chapter2:if}
\end{align}
Listing~\ref{code:example-if} shows the implementation of Equation~\ref{eq:chapter2:if} using a \lstinline[language=c++]{for} loop. The skeleton of the \lstinline[language=c++]{for} loop is identical to the one in Listing~\ref{code:for:loop}, but the \lstinline[language=c++]{if} statement\endnote{\url{https://en.cppreference.com/w/cpp/language/if}} in Line 8 is added to switch between even and odd numbers. The \lstinline[language=c++]{if} statement takes exactly one argument, the so--called condition statement. If the statement is evaluated as \lstinline[language=c++]{true} the code line between \lstinline[language=c++]{if} and \lstinline[language=c++]{else} is selected.  If the statement is evaluated as \lstinline[language=c++]{false} the code line after \lstinline[language=c++]{else} is selected. Multiple lines of codes have to be between curly braces. It is also possible to use \lstinline[language=c++]{else if} after the first \lstinline[language=c++]{if}. \\

\lstinputlisting[language=C++,caption={Computation of the sum from one up to $n$ using the selection statement.\label{code:example-if}},float,floatplacement=h]{ParallelComputationMathExamples/chapter2/lecture1-if.cpp}


The second selection statement is the \lstinline[language=c++]{switch} statement\endnote{\url{https://en.cppreference.com/w/cpp/language/switch}}. For example this statement can be used to execute different code branches depending on a single variable. Listing~\ref{code:example-switch} shows one example to write the name of the color to the standard output. In this case we use a enumeration \lstinline[language=c++]{enum}\endnote{\url{https://en.cppreference.com/w/cpp/language/enum}} to store the colors. The \lstinline[language=c++]{switch} takes one argument and execute the code between the matching \lstinline[language=c++]{case} and the following \lstinline[language=c++]{break}.


\lstinputlisting[language=C++,caption={Computation of the sum from one up to $n$ using the selection statement.\label{code:example-switch}},float,floatplacement=h]{ParallelComputationMathExamples/chapter2/lecture1-switch.cpp}


%----------------------------------------------------------------------------------------
\subsection{Operators}
%----------------------------------------------------------------------------------------
\index{Operators}
For the example in Listing~\ref{code:for:loop} we have seen the operator \lstinline[language=C++]|i<n| which is a so--called comparison operator. Next to the comparison operators, C++ language has following operators\endnote{\url{https://en.cppreference.com/w/cpp/language/operator_precedence}}:
\begin{itemize}
\item Comparison operators, see Table~\ref{sec:1:tab:operator:comp},
\item Arithmetic operators, see Table~\ref{sec:1:tab:operator:arithmetic},
\item Logical operators, see Table~\ref{sec:1:tab:operator:logical}, and
\item Assignment operators, see Table~\ref{sec:1:tab:operator:assign},
\end{itemize}
 logical operators, arithmetic, and assignment.

\begin{table}[p]
\centering
\begin{tabular}{clll}
\toprule
Operator & Name  & Example \\ 
\midrule
\lstinline|==| & Equal to & \lstinline|x==y|\\ 
\lstinline|!=| & Not equal & \lstinline|x!=y|\\ 
\lstinline|>| & Greater than & \lstinline|x > y|\\ 
\lstinline|<| & Less than & \lstinline|x < y|\\ 
\lstinline|>=| & Greater than or equal & \lstinline|x >= y|\\ 
\lstinline|<=| & Less than or equal & \lstinline|x <= y|\\ 
\bottomrule 
\end{tabular} 
\caption{Comparison operators}
\label{sec:1:tab:operator:comp}
\index{Operators!comparison}
\end{table}

\begin{table}[p]
\centering
\begin{tabular}{clll}
\toprule
Operator & Name & Description & Example \\ 
\midrule
\lstinline|+| & Addition & Computes the sum of two values & $2+2=4$ \\ 
\lstinline|-| & Subtraction  & Computes the difference of two values & $5-3=2$ \\ 
\lstinline|/| & Division & Divides two values & $6/2=3$ \\ 
\lstinline|*| & Multiplication & Multiplies two values & $2\times2=4$ \\ 
\lstinline|%| & Modulo &  	Returns the division remainder & \lstinline|2%1=0| \\ 
\lstinline|++| & Increments & Add plus one to the value & \lstinline|1++=2|\\ 
\lstinline|--| & Decrements & Subtract one of the value & \lstinline|1--=0|\\ 
\bottomrule 
\end{tabular} 
\caption{Arithmetic operators}
\label{sec:1:tab:operator:arithmetic}
\index{Operators!arithmetic}
\end{table}


\begin{table}[p]
\centering
\begin{tabular}{clll}
\toprule
Operator & Name & Description & Example \\ 
\midrule
\lstinline|&&| & Logical and & Returns \lstinline|true| if both statements are true  & \lstinline| x > 5 && x < 10| \\ 
\lstinline|||| & Logical or  & Returns \lstinline|true| if one statement is true & \lstinline| x > 5 || y < 10| \\ 
\lstinline|!| & Logical not &  Inverse the statement & \lstinline| !(x > 5 && x < 10)| \\ 
\bottomrule 
\end{tabular} 
\caption{Logical operators}
\label{sec:1:tab:operator:logical}
\index{Operators!logical}
\end{table}


\begin{table}[p]
\centering
\begin{tabular}{clll}
\toprule
Operator & Name & Example & Equivalent  \\ 
\midrule
\lstinline|=| & Assignment &   \lstinline| x = 5| &   \lstinline| x = 5 | \\ 
\lstinline|+=| & Plus equal  & \lstinline| x+= 5|  & \lstinline| x = x + 5 | \\ 
\lstinline|-=| & Minus equal & \lstinline| x-= 5|  & \lstinline| x = x - 5 | \\ 
\lstinline|*=| & Multiplication equal &  \lstinline| x*= 5| & \lstinline| x= x * 5| \\ 
\lstinline|/=| & Division equal &  \lstinline| x/= 5| & \lstinline| x= x / 5| \\ 
\lstinline|%=| & Modulo equal &  \lstinline| x%= 5| & \lstinline| x = x % 5| \\ 
\bottomrule 
\end{tabular} 
\caption{Assignment operators}
\label{sec:1:tab:operator:assign}
\index{Operators!assignment}
\end{table}

\begin{exercise}
Write a small C++ program using selection statements and operators to determine if a given year is a lap year. Following logical statements should be implemented: 
\begin{itemize}
	\item   If year is divided by 4 but not by 100, then it is a leap year.
    \item If year is divided by both 100 and 400, then it is a leap year.
    \item If year is divided by 400, then it is a leap year.
    \item And in all other cases, it is not a leap year.
\end{itemize}
\end{exercise}

%----------------------------------------------------------------------------------------
\section{Functions}
%----------------------------------------------------------------------------------------

%----------------------------------------------------------------------------------------
\section{Structuring source code}
%----------------------------------------------------------------------------------------

%----------------------------------------------------------------------------------------
\subsection{Structs}
%----------------------------------------------------------------------------------------

%----------------------------------------------------------------------------------------
\subsection{Classes}
%----------------------------------------------------------------------------------------


\theendnotes
